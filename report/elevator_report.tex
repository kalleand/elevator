\documentclass[10pt,a4paper]{article}
\usepackage[utf8]{inputenc}
\usepackage[english]{babel}
\usepackage{amsmath}
\usepackage{amsfonts}
\usepackage{amssymb}
\author{Karl Johan Andreasson \and Josef Sunesson}
\title{Green Elevator}
\begin{document}
\maketitle

\begin{abstract}
%TODO 
This report handles the implementation of the Green Elevator project in the course Concurrent Programming. The developed controller handles OK.
\end{abstract}

\section{Introduction}
Elevators are present in almost every apartment complex with more than 1 floor. The controller for these elevators has to receive information about the button presses the users make and schedule the elevators accordingly.

To implement this controller to handle the button presses and also the position updates coming from the elevator is the goal of the project. There are several different aspects that has to be considered when scheduling the different elevator cabins. The most important aspects are to minimize waiting time before a cabin comes to serve the user and to minimize the time spent traveling to the desired floor. Another aspect to take into consideration when implementing a scheduling algorithm for the elevator cabins are power consumption.

In the elevator scheduling problem there are hard deadlines to meet once a press of a button has been made. These deadlines are both to not end up with frustrated end users because the controller of the elevators are too slow to schedule the elevators and to not cause the elevators to miss stopping at a floor because the scheduling algorithm was hard at work scheduling another press of a button from a user. To alleviate these problems and to make the solution future proof if the controller should be used with more powerful hardware, a parallelized implementation is desirable.

Another very important aspect of the elevator controller is that it has to provide a fair scheduling of the jobs it receives. No starvation of a job is allowed. Starvation could happen if a job gets postponed because there are other jobs that get scheduled to be executed before and if these jobs continue to arrive the original job is suffering from starvation.

\section{Program outline}
The implementation of the controller of the elevator was made in the language C++ using monitors with locks from the Pthread library for synchronization. C++ was chosen over Java because of the speed increase C++ produces.

The implementation of the elevator controller uses several threads running simultaneously to minimize the response time of the controller. The threads running are one for every elevator car present, one thread that is reading the commands that are coming from the elevators and then dispatching them to either the scheduler or the thread handling the concerned elevator and one last thread to perform scheduling of the jobs.

The threads communicate through monitors and it also through monitors that synchronization is achieved. The choice of monitors was made because of the abstraction it provides. In the implementation there are several monitors present. One for handling jobs that are to be scheduled, one to handle the communication between the controller and the elevators and one for each elevator car. The need of a monitor for communication between the controller and the elevators is because of the choice of implementing the controller in C++ the communication is performed over TCP to the elevators. This forces the communication to be performed serialized and this is performed using a monitor.

\section{Scheduling}
The scheduler is only concerned with scheduling new requests from users coming to the elevator and pressing either up or down depending on the direction they want to go.

What the scheduler does when it gets such a request is to loop over all the elevators. For each elevator, the scheduler asks what the elevators absolute position relative to the request is. If an elevator cannot handle the request, e.g. when an elevator is going upwards and the new request is to go down, the elevator returns a negative value as its absolute position and thereby indicating that it is not available for the scheduler to use.

In the case the elevator does not return a negative value, the scheduler stores the returned value and the elevator that returned that value. When the scheduler has asked all elevators for its absolute position relative the floor of the request, the elevators are sorted according to which was the closest to the floor of the request. Then the scheduler takes the closest elevator and tries to schedule the request to that elevator. This involves asking the elevator for it absolute position
again to assure that the elevator is still able to handle the request in case that elevator has moved since the scheduler asked it the first time. If the elevator cannot handle the request any longer, the next possible elevator is tried and so on.

If no elevator is able to handle the request to begin with, or if that is the case after checking all the elevators again, the request is added to a FIFO queue of commands that has not been scheduled that the elevators takes requests from regularly. The elevators also make sure to take the first request in the queue in case the elevator turn idle in order to guarantee that every request will eventually be handled.

\section{Handling commands}

\end{document}