\documentclass[10pt,a4paper]{article}
\usepackage[utf8]{inputenc}
\usepackage[english]{babel}
\usepackage{amsmath}
\usepackage{amsfonts}
\usepackage{amssymb}
\author{Karl Johan Andreasson \and Josef Sunesson}
\title{Green Elevator}
\begin{document}
\maketitle

\begin{abstract}
%TODO 
This report handles the implementation of the Green Elevator project in the course Concurrent Programming. The developed controller handles OK.
\end{abstract}

\section{Introduction}
Elevators are present in almost every apartment complex with more than 1 floor. The controller for these elevators has to receive information about the button presses the users make and schedule the elevators accordingly.

To implement this controller to handle the button presses and also the position updates coming from the elevator is the goal of the project. There are several different aspects that has to be considered when scheduling the different elevator car. The most important aspects are to minimize waiting time before a car comes to serve the user, to minimize the time spent travelling to the desired floor. Another aspect to take into consideration when implementing a scheduling algorithm for the elevator cars are power consumption.

In the elevator scheduling problem there are hard deadlines to meet once a press of a button has been made. These deadlines are both to not end up with frustrated end users because the controller of the elevators are too slow to schedule the elevators and to not cause the elevators to miss stopping at a floor because the scheduling algorithm was hard at work scheduling another press of a button from an user. To alleviate these problems and to make the solution future proof if the controller should be used with more powerful hardware a paralleliced implementation is desirable.

Another very important aspect of the elevator controller is that it has to provide a fair scheduling of the jobs it receives. No starvation of a job is allowed. Starvation could happen if a job gets postponed because there are other jobs that get scheduled to be executed before and if these jobs continue to arrive the original job is suffering from starvation.

\section{Program outline}
The implementation of the controller of the elevator was made in the language C++ using monitors with locks from the Pthread library for synchronization. C++ was chosen over Java because of the speed increase C++ produces.

The implementation of the elevator controller uses several threads running simultaneously to minimize the response time of the controller. The threads running are one for every elevator car present, one thread that is reading the commands that are coming from the elevators and then dispatching them to either the scheduler or the thread handling the concerned elevator and one last thread to perform scheduling of the jobs.

The threads communicate through monitors and it also through monitors that synchronization is achieved. The choice of monitors was made because of the abstraction it provides. In the implementation there are several monitors present. One for handling jobs that are to be scheduled, one to handle the communication between the controller and the elevators and one for each elevator car. The need of a monitor for communication between the controller and the elevators is because of the choice of implementing the controller in C++ the communication is performed over TCP to the elevators. This forces the communication to be performed serialized and this is performed using a monitor.

\section{Scheduling}
How to decide which elevator that should be sent to handle the command.

\section{Handling commands}

\end{document}